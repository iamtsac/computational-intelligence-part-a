\documentclass[12pt,a4paper]{article}

\usepackage[utf8]{inputenc}
\usepackage[greek,english]{babel}
\usepackage{float}
\usepackage{sans}
\usepackage{kerkis} 
\usepackage{sans}
\usepackage[LGRgreek]{mathastext}   
\usepackage{graphicx}
\usepackage{enumerate}
\usepackage{enumitem}  
\usepackage{captionf
\usepackage{xcolor}
\usepackage{amsmath}
\usepackage{hyperref,xcolor} 
\hypersetup{
    colorlinks,
    linkcolor={red!50!black},
    citecolor={blue!50!black}, urlcolor={cyan!80!black}
}

\newcommand{\en}{\selectlanguage{english}} 
\newcommand{\tl}{\textlatin} 
\newcommand{\gr}{\selectlanguage{greek}}   
\newcommand{\code}[1]{\texttt{#1}}         
\newcommand{\tsuper}{\textsuperscript} \newcommand{\tsub}{\textsubscript}

\renewcommand{\thesection}{\arabic{section}.} 
\renewcommand{\thesubsection}{\arabic{subsection}.}


\gr \title{{\bf \includegraphics[scale=1.0]{images/up_landscape.jpg} \\ ΤΜΗΜΑ ΜΗΧΑΝΙΚΩΝ ΗΛΕΚΤΡΟΝΙΚΩΝ ΥΠΟΛΟΓΙΣΤΩΝ ΚΑΙ ΠΛΗΡΟΦΟΡΙΚΗΣ  \\ \vspace{3cm}Αναφορά Εργαστηριακής Άσκησης Μέρος Α' \\ Υπολογιστική Νοημοσύνη}}
\author{Κωνσταντίνος Τσάκωνας}
\date{Ακαδημαϊκό έτος 2020-21\\ Χειμερινό Εξάμηνο}

\begin{document}

    \gr \maketitle \newpage

    \section*{\tl{Repository} \gr Κώδικα}
        \underline{\tl{\textbf{\url{https://github.com/iamtsac/computational-intelligence-part-a}}}}

    \section*{Α1. Προεπεξεργασία και Προετοιμασία δεδομένων.} 
        Το σύνολο των δεδομένων μας αποτελείται από $785$ στήλες. Στην πρώτη στήλη του αρχείου \tl{csv} o περιέχεται αριθμός που απεικονίζεται στην εικόνα, ο οποίος είναι το  \tl{label} για το νερωνικό. Οι υπόπολοιπες στήλες αποτέλουν τα \tl{pixels} της εικόνας. Οπότε το νευρωνικό μας θα έχει ως είσοδο τις στήλες που περιέχουν τα \tl{pixels} και θα εκπαιδευτεί στα δεδομένα του \tl{label}. Στο κώδικα εργαζόμαστε ως εξής, αφού φορτώσουμε τα δεδομένα του \tl{csv(train\_csv)} χρησιμοποιόντας το pandas, χωρίζουμε το \tl{y(label)} από το \tl{x(pixels)}. Στη συνέχεια μετασχηματίζουμε τη λίστα της εισόδου σε ένα ώστε να περιέχει $60.000$ μητρώα $28\times28$ διαστάσεων. Τη παραπάνω εργασία εφαρμόζουμε και στα δεδομένα που έχουμε για τον έλεγχο(\tl{train\_csv}). Μέτα από τα παραπάνω θα προκύψει μια δομή δεδομένων \tl{np.array} διαστάσεων ($6000,28,28$) όπου κάθε θέση θα περιέχει στοιχεία της μορφής:
        \begin{equation*}
        X_i = 
        \begin{bmatrix}
            pixel_{1,1} & pixel_{1,2} & \cdots & pixel_{1,28} \\
            pixel_{2,1} & pixel_{2,2} & \cdots & pixel_{2,n} \\
            \vdots      & \vdots      & \ddots & \vdots  \\
            pixel_{28,1} & pixel_{28,2} & \cdots & pixel_{28,28} 
        \end{bmatrix} 
        \end{equation*}



\end{document} 